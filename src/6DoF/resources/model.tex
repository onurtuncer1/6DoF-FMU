\documentclass{article}
\usepackage{amsmath}
\usepackage{amssymb}
\usepackage{amsfonts}
\usepackage{bm}
\usepackage{geometry}
\geometry{margin=1in}

\title{6-DOF Equations of Motion}
\author{}
\date{}

\begin{document}

\maketitle

\section*{Conversation}

\textbf{User:} Convert \(\dot{\mathbf{x}} = \mathbf{f}(\mathbf{x}, \mathbf{u})\) to \(\dot{\mathbf{x}} = \mathbf{f}(\mathbf{x})\) where \(\mathbf{u}\) represents inputs.

\textbf{User:} Example, please?

\section*{Response}

Let's consider an example where the input vector \(\mathbf{u}\) represents external forces and torques acting on a rigid body. We want to convert the system from the form \(\dot{\mathbf{x}} = \mathbf{f}(\mathbf{x}, \mathbf{u})\) to \(\dot{\mathbf{x}} = \mathbf{f}(\mathbf{x})\), where \(\mathbf{u}\) is embedded within \(\mathbf{x}\).

\subsection*{1. Initial Form: \(\dot{\mathbf{x}} = \mathbf{f}(\mathbf{x}, \mathbf{u})\)}

Given:
\[
\mathbf{x} = \begin{pmatrix} \mathbf{r} \\ \mathbf{v} \\ \mathbf{q} \\ \boldsymbol{\omega} \end{pmatrix}
\quad \text{and} \quad
\mathbf{u} = \begin{pmatrix} \mathbf{F} \\ \mathbf{M} \end{pmatrix}
\]
where:
\begin{itemize}
    \item \(\mathbf{r}\) represents the position vector.
    \item \(\mathbf{v}\) represents the velocity vector.
    \item \(\mathbf{q}\) represents the quaternion orientation.
    \item \(\boldsymbol{\omega}\) represents the angular velocity vector.
    \item \(\mathbf{F}\) represents the external force vector.
    \item \(\mathbf{M}\) represents the external moment (torque) vector.
\end{itemize}

The equations of motion can be written as:
\[
\dot{\mathbf{x}} = \mathbf{f}(\mathbf{x}, \mathbf{u}) =
\begin{pmatrix}
\mathbf{v} \\
\frac{\mathbf{F}}{m} \\
\frac{1}{2} \mathbf{\Omega}(\boldsymbol{\omega}) \mathbf{q} \\
\mathbf{I}^{-1} \left( \mathbf{M} - \boldsymbol{\omega} \times (\mathbf{I} \boldsymbol{\omega}) \right)
\end{pmatrix}
\]

\subsection*{2. Embedding \(\mathbf{u}\) into \(\mathbf{x}\)}

To convert \(\mathbf{f}(\mathbf{x}, \mathbf{u})\) to \(\mathbf{f}(\mathbf{x})\), we can include \(\mathbf{u}\) (the forces and moments) directly as part of the state vector \(\mathbf{x}\). We'll create an extended state vector \(\mathbf{x}'\) that includes these inputs:
\[
\mathbf{x}' = \begin{pmatrix} \mathbf{r} \\ \mathbf{v} \\ \mathbf{q} \\ \boldsymbol{\omega} \\ \mathbf{F} \\ \mathbf{M} \end{pmatrix}
\]
This extended state vector now has 19 components instead of 13.

\subsection*{3. New ODE System: \(\dot{\mathbf{x}} = \mathbf{f}(\mathbf{x})\)}

We rewrite the ODE system considering the extended state vector \(\mathbf{x}'\):
\[
\dot{\mathbf{x}}' = \mathbf{f}(\mathbf{x}')
\]
where:
\[
\mathbf{f}(\mathbf{x}') =
\begin{pmatrix}
\dot{\mathbf{r}} \\ \dot{\mathbf{v}} \\ \dot{\mathbf{q}} \\ \dot{\boldsymbol{\omega}} \\ \dot{\mathbf{F}} \\ \dot{\mathbf{M}}
\end{pmatrix}
=
\begin{pmatrix}
\mathbf{v} \\
\frac{\mathbf{F}}{m} \\
\frac{1}{2} \mathbf{\Omega}(\boldsymbol{\omega}) \mathbf{q} \\
\mathbf{I}^{-1} \left( \mathbf{M} - \boldsymbol{\omega} \times (\mathbf{I} \boldsymbol{\omega}) \right) \\
\mathbf{0} \\
\mathbf{0}
\end{pmatrix}
\]

\textbf{Explanation:}
\begin{itemize}
    \item \(\mathbf{F}\) and \(\mathbf{M}\) are now treated as part of the state vector and assumed constant in this simple example. Thus, their time derivatives \(\dot{\mathbf{F}} = \mathbf{0}\) and \(\dot{\mathbf{M}} = \mathbf{0}\).
    \item The state vector \(\mathbf{x}'\) is now 19-dimensional, and the function \(\mathbf{f}(\mathbf{x}')\) incorporates all the necessary dynamics.
\end{itemize}

\subsection*{4. Implementing in Eigen}

Here is how you can implement this in C++ using the Eigen library:

\begin{verbatim}
#include <Eigen/Dense>

using StateVector = Eigen::Matrix<double, 19, 1>;

// Define aliases for convenience
using Vector3d = Eigen::Vector3d;
using Quaterniond = Eigen::Quaterniond;

StateVector computeStateDerivative(const StateVector& x_prime, const Eigen::Matrix3d& inertia, double mass) {
    StateVector xdot_prime;

    // Extract components from the extended state vector
    Vector3d r = x_prime.segment<3>(0);  // Position (r_x, r_y, r_z)
    Vector3d v = x_prime.segment<3>(3);  // Velocity (v_x, v_y, v_z)
    Quaterniond q(x_prime(6), x_prime(7), x_prime(8), x_prime(9));  // Quaternion (q_0, q_1, q_2, q_3)
    Vector3d omega = x_prime.segment<3>(10);  // Angular velocity (omega_x, omega_y, omega_z)
    Vector3d F = x_prime.segment<3>(13);  // Force (F_x, F_y, F_z)
    Vector3d M = x_prime.segment<3>(16);  // Moment (M_x, M_y, M_z)

    // Compute the derivatives
    Vector3d rdot = v;  // Position derivative is the velocity
    Vector3d vdot = F / mass;  // Acceleration
    Quaterniond qdot = 0.5 * Quaterniond(0, omega.x(), omega.y(), omega.z()) * q;  // Quaternion derivative
    Vector3d omegadot = inertia.inverse() * (M - omega.cross(inertia * omega));  // Angular velocity derivative

    // Populate the state derivative vector
    xdot_prime.segment<3>(0) = rdot;  // rdot
    xdot_prime.segment<3>(3) = vdot;  // vdot
    xdot_prime.segment<4>(6) = Eigen::Vector4d(qdot.w(), qdot.x(), qdot.y(), qdot.z());  // qdot
    xdot_prime.segment<3>(10) = omegadot;  // omegadot
    xdot_prime.segment<3>(13) = Vector3d::Zero();  // Force is constant
    xdot_prime.segment<3>(16) = Vector3d::Zero();  // Moment is constant

    return xdot_prime;
}
\end{verbatim}

\textbf{Summary:}
\begin{itemize}
    \item \texttt{StateVector} is now 19x1, with the forces and moments included in the state.
    \item \texttt{computeStateDerivative} now returns the state derivative without needing external inputs because all necessary variables are embedded within the state vector.
\end{itemize}

This approach effectively removes the input vector \(\mathbf{u}\) by embedding it into the extended state vector \(\mathbf{x}'\), making the system fully self-contained.

\end{document}
